% vim: spell spelllang=pt formatoptions+=tcaw textwidth=78 nonu foldcolumn=0

\documentclass[12pt]{article}
\usepackage[brazil]{babel}
\usepackage[utf8]{inputenc}
\usepackage[T1]{fontenc}
\usepackage{sbc-template}
\usepackage{graphicx,url}

     
\sloppy

\title{Transições Interrede Independentes do Meio Físico de Acesso}

\author{Higor Eurípedes P. F. A.\inst{1}, Cláudio de C. Monteiro\inst{1}}

\address{Instituto Federal de Educação, Ciência e Tecnologia do Tocantins (IFTO)\\
  77.021-090 -- Palmas -- TO -- Brazil
  \email{heuripedes@gmail.com, ccm@ifto.edu.br}
}

\date{Novembro, 2011}

\begin{document}

\maketitle

\begin{abstract}

This document describes, superficially, the features of the IEEE 802.21 
standard for \textit{Media Independent Handover}.

\end{abstract}
     
\begin{resumo}

Este documento descreve, superficialmente, as características do padrão
IEEE 802.21 para transições interrede independentes do meio físico de acesso.

\end{resumo}


\section{Introdução}

A crescente procura por mobilidade por parte da população, impulsionou a venda 
de dispositivos móveis assim como o celular e os computadores portáteis. Cada 
vez mais exigentes, os usuários demandam dispositivos que os permitam proceder 
com suas atividades onde e como lhe convir.

Para atender à vontade do mercado e da população em geral, padrões como o
IEEE 802.11\cite{ieee:2007:80211} e o 802.16\cite{ieee:2009:80216} foram 
desenvolvidos.  Entretanto, a evolução dos dispositivos móveis introduziu o 
conceito de multiacesso, uma característica que incrementou a experiência dos 
usuários, mas explicitou que as tecnologias de conectividade possuíam uma 
deficiência: a transição de uma rede para outra era incômoda e causava quedas 
de disponibilidade de certos serviços. Segundo \cite{piri:2009}, isto 
acontecia, parcialmente, por falta de padronização dos mecanismos de transição 
em uso.  Como resposta a estas deficiências, deu-se inicio ao desenvolvimento 
de um padrão aberto para a transição interrede independente do meio físico de 
acesso: o padrão IEEE 802.21 \cite{ieee:2008:80221}.

\section{O padrão IEEE 802.21}

O padrão IEEE 802.21 documenta mecanismos que possibilitam a transição entre 
redes heterogêneas.  Em \cite{idigital:2009}, o padrão é caracterizado como 
uma abstração da camada de enlace, oferecendo às camadas superiores um acesso 
comum independente da tecnologia de enlace empregada.

É mencionado em \cite{ieee:2008:80221}, que o objetivo principal do padrão é 
facilitar o processo de transição entre redes IEEE 802 - tornando-o 
independente do meio de transmissão utilizado - e permitir a conectividade 
ininterrupta dos dispositivos garantindo uma experiência de continuidade de 
conexão para o usuário.  Embora tenha foco na transição entre redes IEEE 802 
heterogêneas, o padrão também poderá ter seus mecanismos utilizados para 
efetuar transições homogêneas ou entre redes IEEE 802 e redes não-IEEE 802.

Segundo \cite{piri:2009}, o elemento principal do padrão é a \textit{Media 
Independent Handover Function} (MIHF). Esta função é uma entidade lógica, 
localizada entre as camadas 2 e 3, que tem a tarefa de assistir transições dos 
\textit{Mobile Nodes} (MN), dispositivos com capacidade para acessar várias 
redes simultâneamente. Em \cite{kimhun:2010}, é afirmado que a MIHF oferece 
uma abstração das camadas inferiores sob a forma de interface unificada, 
visando assistir as aplicações das camadas superiores durante o processo de 
\textit{handover}. Esta interface é composta de três serviços disponibilizados 
aos nós da rede:

\begin{enumerate}

	\item \textit{Media Independent Event Service} (MIES) -- fornece 
	classificação, filtragem e notificação de eventos correspondentes à 
	alterações das propriedades das camadas inferiores. O serviço também é 
	capaz notificar as camadas superiores de eventos que irão ocorrer.

	\item \textit{Media Independent Command Service} (MICS) -- permite que 
	camadas superiores sejam capazes de reconfigurar ou selecionar 
	conexões por meio de comandos de transição.

	\item \textit{Media Independente Information Service} (MIIS) --  
	fornece detalhes sobre serviços disponíveis na rede atual e nas 
	vizinhas.
	
\end{enumerate}

A transmissão e recepção de comandos, eventos e informações entre os serviços 
e os clientes MIH, são feitos utilizando o protocolo \textit{Media Independent 
Handover Protocol} (MIHP). Segundo \cite{ieee:2008:80221}, este protocolo 
permite a troca de mensagens MIH entre as MIHFs das entidades ao nível das 
camadas dois e três. A comunicação pode ser feita de forma síncrona ou 
assíncrona, utilizando pares de requisição e resposta.

O MIHP abrange tecnologias IEEE como IEEE 802.3 e IEEE 802.16, mas também 
tecnologias não-IEEE como \textit{3GPP} e \textit{3GPP2}. O suporte a estas 
redes é oferecido por modelos de referência de implementação, documentados no 
próprio padrão de MIH.

\subsection{As transições}

Como apresentado anteriormente, o padrão 802.21 possui foco nas transições 
verticais, isto é, transições entre redes heterogêneas. Estas transições são 
realizadas de modo a garantir um uso otimizado dos recursos da rede e dos 
dispositivos, além de promover a continuidade dos serviços oferecidos aos 
usuários.

Transições que obedecem o padrão IEEE 802.21 podem ser motivadas por eventos 
propagados pelo MIES ou por comandos enviados ao MICS. O MIIS fica responsável 
por disponibilizar às camadas superiores informações suficientes sobre a 
situação da rede alvo, de modo a permitir a escolha que apresente menor custo.  
Detalhes específicos sobre os procedimentos da operação ficam sob a 
responsabilidade do fornecedor, segundo \cite{stein:2006}.

Para exemplificar o trabalho dos serviços durante um \textit{handover}, 
pode-se considerar a seguinte situação adaptada de \cite{kimhun:2010}: 

\begin{enumerate}

	\item Um determinado local é coberto por duas redes, uma WiFi (rede A) 
	e outra WiMAX (rede B), e, no momento, existem alguns dispositivos 
	clientes (C) conectados à rede A;

	\item O administrador da rede A decide efetuar a manutenção no único 
	\textit{access point} (\textit{Point of Attachment} ou PoA) da rede e, 
	utilizando um programa de gerência com suporte a MIH e MICS, envia à 
	MIHF de seus clientes uma \textit{MIH\_Net\_HO\_Candidate\_Query 
	request} para iniciar a transição e sugerir pontos de acesso 
	alternativos. Os clientes respondem com uma 
	\textit{MIH\_Net\_HO\_Candidate\_Query response} informando os pontos 
	de acesso  e conexões preferidas;

	\item O PoA da rede A solicita informações adicionais aos MIIS dos 
	PoAs sugeridos pelo cliente por meio de uma 
	\textit{MIH\_N2N\_HO\_Query\_Resource request}, e os mesmos respondem 
	com uma \textit{MIH\_N2N\_HO\_Query\_Resource response}. Com esta 
	informação o PoA atual determina qual PoA receberá os clientes, no 
	caso, o PoA da rede B.

	\item Uma \textit{MIH\_Net\_HO\_Commit request} contendo o PoA de destino 
	é enviada aos clientes para que efetuem a transição. O PoA atual 
	notifica os PoAs vizinhos que a transição irá ocorrer utilizando uma 
	\textit{MIH\_N2N\_HO\_Commit request} que informa qual cliente e quais 
	PoAs estão envolvidos. Os clientes e o PoA alvo ao concluirem a 
	operação de \textit{handover} respondem ao PoA inicial com uma 
	\textit{MIH\_Net\_HO\_Commit response} e \textit{MIH\_N2N\_HO\_Commit 
	response}, respectivamente.

	\item Adicionalmente, o PoA alvo e o cliente podem enviar mensagens 
	\textit{MIH\_N2N\_HO\_Complete} ou \textit{MIH\_Net\_HO\_Complete} ao PoA 
	inicial para confirmar a conclusão da transição.

\end{enumerate}

A situação mais comum, vivida em ambientes que implementam MIH, é a que ocorre 
sem intervenção, por meio de eventos propagados pelo MIES dos MN. Tomando o 
exemplo anterior como ponto de partida, pode-se substituir a intervenção do 
administrador de redes pela degradação de sinal, causada pela movimentação do 
dispositivo. Neste momento, o MIES do cliente dispara o evento 
\textit{MIH\_Link\_Going\_Down} para que o PoA associado solicite informações 
do estado dos PoAs adjacentes.


\bibliographystyle{sbc}
\bibliography{minuta}

\end{document}
